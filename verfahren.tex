\chapter{Verfahren}\label{ch:verfahren}


\section{Standards aus verschiedenen Sprachen}\label{sec:standards-aus-verschiedenen-sprachen}

\subsection{Java}\label{subsec:java}
\textcolor{blue}{
    Die Standardbibliothek bietet seit Java-Version 1.4 ein Logging-Framework.
    Dieses Paket enthält unter anderem einen Logger.
    Dieser kann durch den Import \enquote{\texttt{import java.util.logging.Logger}} genutzt werden.
    Um einen Logger zu nutzen, muss er auch initialisiert werden.
    Hierfür wird der Logger mit dem Klassennamen als Parameter, der nutzenden Klasse, erstellt oder der entsprechende Logger gefunden:
    \\
    \hspace*{10mm}
    \texttt{\textbf{private static final} Logger LOGGER =
        \\
        \hspace*{30mm}Logger.getLogger(classname.\textbf{class}.getName());}
    \\
    Um eine Nachricht mit dem Logger zu erstellen, muss die Methode entsprechend zu dem Loglevel genutzt werden.
    Für eine informelle Nachricht wird die Methode \texttt{info()} genutzt, beispielsweise:
    \\
    \hspace*{10mm}
    \texttt{LOGGER.info(\enquote{Loggername: } + LOGGER.getName());}
    \\
    Hierbei muss der Logger mindestens das Level INFO konfiguriert haben, welches als Standardebene eingestellt ist.
    Hierdurch entsteht folgender Log mit path als Variable für den Package- und Klassenname:
    \\
    \hspace*{10mm}
    \texttt{Jun 23, 2022 1:19:30 PM path main INFO: Loggername: path}
    \\
    Die Nachrichten können auch über eine allgemeinere Methode erstellt werden.
    Diese nennt sich \texttt{log()} und nimmt als Parameter das Loglevel und die Nachricht an.
    Sowie ein Objekt, welches protokolliert werden soll.
    Es können Objekt wie Exceptions gespeichert werden.
    Um diese Methode nutzen zu können muss ein weiterer Import gesetzt sein.
    Denn es werden die statischen Einträge der Levelklasse genutzt.
    \\
    \hspace*{10mm}
    \texttt{LOGGER.log(Level.SEVERE, \enquote{Eine Exception ist aufgetreten}, ex)}
    \\
    Hierbei entsteht eine Nachricht in gleichem Format:
    \\
    \hspace*{10mm}
    \texttt{Jun 23, 2022 1:19:31 PM path main SEVERE: Eine Exception
    \\
    \hspace*{30mm} ist aufgetreten.}
    \\
    \\
    Vorher wurde bereits Log4J, welches im Verlauf erläutert wird, entwickelt, weshalb ein starker Vergleich gezogen wurde.
    Der Standard aus Java ist nicht kompatibel mit Log4J.
    Des Weiteren ist es nicht so leistungsstark wie Log4J.
    \\
    Das Logging aus Java hat verschiedene Loglevel, die den Schwierigkeitsgrad des Logs zeigen.
    Diese heißen SEVERE, WARNING, INFO, CONFIG, FINE, FINER und FINEST
    Je nach Loglevel werden die Nachrichten protokolliert oder ignoriert.
    Falls der Logger mit dem Loglevel WARNING konfiguriert wurde, wird die Nachricht mit Level INFO ignoriert.
    Daher kann unnötiger Aufwand entstehen, falls eine Nachricht, die aufwändig aufgebaut wird, nicht geschrieben wird, jedoch vorher schon eine Variable ausgewertet hat.
    Im diesen Aufwand zu verringern wurde die Methode boolean isLoggable(Level level) der Bibliothek hinzugefügt.
    Diese wertet den Ausdruck in der Log-Nachricht erst aus, falls die Nachricht geloggt werden darf.
    Eine weitere Möglichkeit den Aufwand zu verringern gibt es seit Java-Version 8.
    Hier wurden Lambda-Ausdrücke eingeführt, um dne Programmcode in einem Block zusammen erst auszuführen, falls das Logging-Framework den Block ausführen soll.
    \\
    Das Logging-Framework enthält als Komponente einen Handler, der die geloggten Nachrichten an einen bestimmten Zielort speichert.
    Der KonsolenHandler schreibt die geloggten Nachrichten in die Konsole.
    Der FileHandler speicher die Logs in Dateien.
    Es gibt noch weitere Handler, die für das Speichern der Daten genutzt werden können.
    \\
    Eine weitere Komponente des Logging-Frameworks ist ein Formatter.
    Jedem Handler ist ein Formatter zugeordnet.
    Es wird unterschieden zwischen SimpleFormatter und XMLFormatter.
    Der SimpleFormatter wird genutzt, um die Logs für den Menschen einfach lesbar zu machen.
    Für den FileHandler ist ist der XMLFormatter standardisiert.
}\autocite{rheinwerk, baeldung, examples}

\subsection{JavaScript}\label{subsec:javascript}
\textcolor{blue}{
    Bei der Entwicklung einer Webanwendung können die Fehlermeldung in der Konsole ausgegeben werden.
    Bei veröffentlichtem Code ist die Konsole nicht verfügbar.
    Daher wird die Möglichkeit genutzt, die Fehlermeldungen bei dem Client zu sammeln und an einen Server zu senden, wo diese Logs auszuwerten sind.
    Auf der Konsole können mehrere Methoden verwendet werden, die den verschiedenen Logleveln entsprechen.
    Für JavaScript gibt es vier Loglevel.
    Das Geringste ist für Klartext und nutzt die Methode \enquote{log()}.
    Das nächste Level ist Info mit der Methode \enquote{info()}.
    Außerdem gibt es wie bei anderen Log-Frameworks Warn und Error mit \enquote{warn()} und \enquote{error()}.
    Mit diesen Aufrufen werden die Nachrichten entsprechend unterlegt, entsprechend dem Level.
    Hierfür werden die Methoden wie folgt aufgerufen:
    \\
    \hspace*{10mm}
    \texttt{console.log(msg)}
}\autocite{stackify}

\subsection{Python}\label{subsec:python}
\textcolor{blue}{
}
\subsection{C++}\label{subsec:c++}


\section{Logging}\label{sec:logging-verfahren}

\textcolor{blue}{
    Bis die Standardbibliothek von Java um einen Logger ergänzt wurde, wurde bereits das Log4J entwickelt.
    Log4J ist ein Open Source Projekt.
    Es wird durch eine Apache Software License verbreitet.
    Log4J ist eine populäre API, weshalb sie in mehrere Sprachen übersetzt wurde.
    Beispielsweise wurde sie in C++ übersetzt, dort heißt die API Log4cxx.
    \\
    Die zentralen Begriffe der API sind Logger, Level, Appender und Layout.
    Logger entspricht dem Zentrum des Interesse.
    Dieser kümmert sich um die Logs.
    Erstellt sie, falls diese dem richtigen Level entsprechen.
    Level ist das Loglevel.
    Der Schwierigkeitsgrad des Ereignisses.
    Falls der Logger das entsprechende Level des Ereignisses hat, soll die Nachricht in das Protokoll geschrieben werden.
    Der Appender kümmert sich um das Anhängen der Ausgabe.
    Falls die Ereignisse in Dateien geschrieben werden, sorgt der Appender dafür, dass die vorher geschriebenen Ereignisse nicht geschrieben werden.
    Layout ist das Format, indem die Nachrichten festgehalten werden, beispielsweise HTML oder XML.
    Das Logging mit Log4J lässt sich über Java Anweisungen konfigurieren.
    Es kann aber auch eine Konfigurationsdatei verwendet werden.
    Mittlerweile ist diese Bibliothek veraltet und dient nur als Grundlage für moderne Nachfolger.
    \\
    \\
    Eine Verbesserung dieses Logging-Frameworks ist Log4J2.
    Die größte Verbesserung ist das asynchrone Protokollieren.
    Hierdurch wird das Multithreading möglich und der Logger bleibt in keiner Deadlock, wie es in Log4J1 war.
    Bei eigenständigen Anwendungen ist Log4J2 \enquote{müllfrei}, weshalb der Garbage Collector den Logger nicht entfernt.
    Durch das Plugin-System ist Log4J2 einfach zu erweitern.
    Beispielsweise einen neuen Appender hinzuzufügen ist einfach, da Log4J nicht verändert werden muss.
    Log4J2 unterstützt für den Appender gewünschte Formate.
    Zum Beispiel Logback kann nur ein bestimmtes Format mit dem Appender transportieren.
}\autocite{rheinwerk, baeldung, log4j2}


\section{Tracing}\label{sec:tracing-verfahren}

%\subsection{Jaeger}\label{subsec:jaeger}


\section{Monitoring}\label{sec:monitoring-verfahren}

%\subsection{Grafana}\label{subsec:grafana}
%\subsection{Prometheus}\label{subsec:prometheus}
%\subsection{Argos}\label{subsec:argos}


\section{Weiterentwicklungen}\label{sec:weiterentwicklungen}


\section{Performance}\label{sec:performance}