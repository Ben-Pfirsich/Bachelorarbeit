\chapter{Einleitung} \label{sec:introduction}

In diesem Kapitel wird die Nutzung einiger \LaTeX{}-Funktionen und Funktionen aus verschiedenen Paketen vorgeführt.

\section{Referenzen} \label{sec:references}

In den folgenden Abschnitten sind Umsetzungen von Zitierungen und Verweise zu finden.

\subsection{Zitierungen}

Quellenangaben werden am Besten mit \enquote{autocite} durchgeführt \autocite{Mann1947}, Befehle um weitere Informationen aus der Quelle auslesen zu können sind ebenfalls vorhanden: \textcite{Mann1947}, \citeauthor{Mann1947}, \citetitle{Mann1947}.

\subsection{Verweise} \label{sec:references:links}

Verweise auf Kapitel, Abschnitte, Bilder und Tabellen werden mit dem \enquote{autoref} Befehl eingefügt: \autoref{sec:introduction}, \autoref{sec:references}, \autoref{sec:references:links}, \autoref{fig:results:spindle-detection-u-net}, \autoref{tab:results:f1-scores}.
Ein benutzerdefinierter Text kann hierbei wie folgt eingefügt werden: Im \hyperref[sec:summary]{letzten Kapitel} wird eine Zusammenfassung durchgeführt.

\section{Akronyme}

Akronyme wie \gls{cnn} werden so benutzt und besitzen unterschiedliche Ausgaben bei der ersten Nutzung und bei darauffolgenden Verwendungen, bei denen nur die Kurzform (\gls{cnn}) genutzt wird.

\section{Weitere Funktionen}

Zahlen, Einheiten und Kombinationen daraus können besonders gut mit dem \textit{siunitx} Paket umgesetzt und konsistent formatiert werden.
So kann \zB{} eine Liste von Zahlen als \numlist{1.25; 2.5; 5} und ein Frequenzbereich als \SIrange{11}{16}{\Hz} (optional mit anderer Formatierung, \zB{} \SIrange[range-phrase=--, range-units=single]{11}{16}{\Hz}) angegeben werden.
Auch Prozentangaben sind möglich: \SI{80}{\percent}.
